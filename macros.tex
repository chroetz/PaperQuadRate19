% ------------------------------------------------------------------------------
% ------------------------------------------------------------------------------
%
%
\newcommand*{\todo}[1]{[\textbf{\color{red} Todo:} \textit{#1}]}
\newcommand*{\note}[1]{[\textbf{\color{blue} Note:} \textit{#1}]}
%
%
%
% Math modes can be tested for: \ifmmode is true in display and non-display math mode, and \ifinner is true in non-display mode, but not in display mode.
%\edef\modedef{This macro was defined in \ifvmode vertical\else \ifmmode math \else horizontal\fi\fi' mode}
%
%
\renewcommand{\subset}{\subseteq}
%
%
% brackets
\newcommand{\lcb}{\left\lbrace} % '{' Left Curly Bracket
\newcommand{\rcb}{\right\rbrace} % '}' Right Curly Bracket
\newcommand{\cb}[1]{\lcb #1 \rcb} % Curly Brackets
\newcommand{\cbOf}[1]{\mathopen{}\lcb #1 \rcb\mathclose{}} % Curly Brackets
\newcommand{\lab}{\left[} %'[' Left square/Angular Bracket
\newcommand{\rab}{\right]} %']' Right square/Angular Bracket
\newcommand{\ab}[1]{\lab #1 \rab} % Brackets
\newcommand{\abOf}[1]{\!\ab{#1}} % Brackets
\newcommand{\lb}{\left(} %'(' Left (round) Bracket
\newcommand{\rb}{\right)} %')' Right (round) Bracket
\newcommand{\br}[1]{\lb #1 \rb} % Brackets
\newcommand{\brOf}[1]{\!\br{#1}} % Brackets
\newcommand{\abs}[1]{\left| #1 \right|} % Brackets
%
% MathOperators
\newcommand*{\E}{\mathbb{E}} % expectation
\newcommand*{\ent}{\mathsf{Ent}} % entropy
\newcommand*{\V}{\mathbb{V}} % expectation
\newcommand*{\cov}{\mathbb{COV}} % expectation
\let\Pr\relax% Set equal to \relax so that LaTeX thinks it's not defined
\newcommand*{\Pr}{\mathbb{P}} % probability
\newcommand*{\powset}{\mathfrak{P}} % power set
% *Of (inserts brackets, mathopen/-close used for correct spacing)
\newcommand{\sizedMid}[2]{#1 \, \kern-\nulldelimiterspace\mathopen{}\left| \vphantom{#1}\,#2\right.\mathclose{}\kern-\nulldelimiterspace}
\newcommand{\EOf}[1]{\E\abOf{#1}}
\newcommand{\Eof}[1]{\E[#1]}
\newcommand{\entOf}[1]{\ent\brOf{#1}}
\newcommand{\entof}[1]{\ent(#1)}
\newcommand{\Vof}[1]{\V[#1]}
\newcommand{\VOf}[1]{\V\abOf{#1}}
\newcommand{\covof}[1]{\cov[#1]}
\newcommand{\ECondOf}[2]{\EOf{\sizedMid{#1}{#2}}}
\newcommand{\ECondof}[2]{\E[#1\mid #2]}
\newcommand{\PrOf}[1]{\Pr\mathopen{}\lb #1 \rb\mathclose{}}
\newcommand{\Prof}[1]{\Pr(#1)}
\newcommand{\PrCondOf}[2]{\PrOf{\sizedMid{#1}{#2}}}
\newcommand{\PrCondof}[2]{\Pr(#1\mid #2)}
\newcommand{\powsetOf}[1]{\powset\mathopen{}\lb #1 \rb\mathclose{}}
\newcommand{\eventAnd}{,\,} %eventAnd
\newcommand{\setByEle}[2]{\cb{\sizedMid{#1}{#2}}}
\newcommand{\setByEleInText}[2]{\{#1 \mid #2\}}
\DeclareMathOperator{\diam}{\mathsf{diam}}
\DeclareMathOperator{\card}{\mathsf{card}}
\DeclareMathOperator{\ball}{\mathrm{B}}
\newcommand{\id}{\mathrm{id}}
\newcommand{\restrict}[1]{_{\mkern 1mu \vrule height 2ex\mkern2mu {#1}}}
%
\providecommand\given{} % so it exists
\newcommand\SetSymbol[1][]{
	\nonscript\,#1\vert \allowbreak \nonscript\,\mathopen{}}
\DeclarePairedDelimiterX\Set[1]{\lbrace}{\rbrace}%
{ \renewcommand\given{\SetSymbol[\delimsize]} #1 }
%
%\newcommand\myEx[1][]{\nonscript\,#1\vert \allowbreak \nonscript\,\mathopen{}}
%\DeclarePairedDelimiterX{\myexpectarg}[1]{\lbrace}{\rbrace}{\renewcommand\given{\myEx[\delimsize]} #1 }
%
\newcommand{\Ex}{\E\expectarg}
\DeclarePairedDelimiterX{\expectarg}[1]{[}{]}{%
	\ifnum\currentgrouptype=16 \else\begingroup\fi
	\activatebar#1
	\ifnum\currentgrouptype=16 \else\endgroup\fi
}
\newcommand{\innermid}{\nonscript\;\delimsize\vert\nonscript\;}
\newcommand{\activatebar}{%
	\begingroup\lccode`\~=`\|
	\lowercase{\endgroup\let~}\innermid 
	\mathcode`|=\string"8000
}
%
% cardinality of a set: |A|
\newcommand{\cardOf}[1]{\left\vert{#1}\right\vert}
\newcommand{\cardof}[1]{\vert{#1}\vert}
%
% image of a map
\newcommand*{\im}{\mathrm{im}}
% 
\newcommand*{\mc}[1]{\mathcal{#1}}
\newcommand*{\mb}[1]{\mathbb{#1}}
\newcommand*{\mr}[1]{\mathrm{#1}}
\newcommand*{\ms}[1]{\mathsf{#1}}
\newcommand*{\mo}[1]{\mathbf{#1}}
\newcommand*{\mf}[1]{\mathfrak{#1}}
%
% natural numbers
\newcommand{\N}{\mathbb{N}}
\newcommand{\Nn}{\mathbb{N}_0}
% real numbers
\newcommand{\R}{\mathbb{R}}
\newcommand{\Rp}{\R_+} % [0, \infty)
\newcommand{\Ra}{\bar\R} % extended reals

% borel
\DeclareMathOperator{\borel}{\mathcal{B}}
\newcommand{\borelof}[1]{\borel(#1)}
%
% C
\newcommand{\C}[2]{\mathcal{C}^{#1}\br{#2}}
% 
% derivative
\newcommand{\derive}[1]{#1^\prime}
%
\newcommand{\transpose}{\!^\top\!}
\newcommand{\tr}{\transpose}
%
\newcommand{\pr}{^\prime}
\newcommand{\prr}{^{\prime\prime}}
\newcommand{\prrr}{^{\prime\prime\prime}}
% 
% [0, 1]
\newcommand{\normIntervall}{\lab 0, 1 \rab} %
%
% Integral
%\newcommand{\intOf}[4]{\int_{#1}^{#2} \! #3 \mathrm{d}#4} %
\def\integral from #1to #2of #3by #4;{\int_{#1}^{#2} \! #3 \mathrm{d}#4} %
\def\integralMeasure in #1of #2by #3of #4;{\int_{#1} \! #2{#4} #3{\mathrm{d}#4}} %
%
% Functiondefintion
%\newcommand{\mapping}[3]{#1 \colon #2 \rightarrow #3}
\def\mapping #1from #2to #3;{#1 \colon #2 \rightarrow #3}
\def\mappingDef #1from #2to #3maps #4to #5;{#1 \colon #2 \rightarrow #3,\ #4 \mapsto #5}
%
% Sequence (Folge)
\def\seq #1by #2;{\br{#1}_{#2\in\N}}
\def\seqInText #1by #2;{(#1)_{#2\in\N}}
%
%
\newcommand{\innerProduct}[2]{\left\langle#1\,,\, #2\right\rangle}
\newcommand{\ip}[2]{\innerProduct{#1}{#2}}
%
%
\newcommand{\lebesgue}{\mathcal{L}}
\newcommand{\lebesguePow}[1]{\lebesgue^{#1}}
\newcommand{\lebesgueOf}[1]{\lebesgue\brOf{#1}}
%
%
\newcommand{\invert}[1]{#1^{-1}}
%
\newcommand{\sgn}{\mathsf{sgn}} % signum
\newcommand{\supp}{\overline{\mathsf{supp}}} % support
\newcommand{\median}{\mathsf{median}} % median
%
\newcommand{\dl}{\mathrm{d}}
%
\newcommand\independent{\protect\mathpalette{\protect\independenT}{\perp}}
\def\independenT#1#2{\mathrel{\rlap{$#1#2$}\mkern2mu{#1#2}}}
%
%
\def\converges for #1to #2;{\xrightarrow{#1} #2}
\def\convergesAlmostSurely for #1to #2;{\xrightarrow{#1}_{\mathsf{fs}} #2}
\def\convergesInProbability for #1to #2;{\xrightarrow{#1}_{\mathsf{p}} #2}
\def\convergesInL #1for #2to #3;{\xrightarrow{#2}_{\lebesguePow{#1}} #3}
\newcommand{\as}{\mathsf{a.s.}}
%
% Indicator
\newcommand{\ind}{\mathds{1}}% indicator function
\newcommand{\indOf}[1]{\ind_{\!#1}}% 
\newcommand{\indOfOf}[2]{\ind_{\!#1}\!\brOf{#2}}% 
\newcommand{\indOfEvent}[1]{\indOf{\cbOf{#1}}}% 
\newcommand{\indOfEventOf}[2]{\indOf{\cbOf{#1}}\!\brOf{#2}}% 
%
% complement of a set
%\newcommand{\complOf}[1]{#1^\complement}% 
%\newcommand{\complOf}[1]{\inSpace\setminus#1}% 
\newcommand{\compl}{^\mathsf{c}}% 
%
% Landau symbols
\DeclareMathOperator{\landauO}{\mathcal{O}}%
\newcommand{\landauOOf}[1]{\landauO\brOf{#1}}% 
\DeclareMathOperator{\landauOmega}{\Omega}%
\newcommand{\landauOmegaOf}[1]{\landauOmega\brOf{#1}}% 
\DeclareMathOperator{\landauTheta}{\Theta}%
\newcommand{\landauThetaOf}[1]{\landauTheta\brOf{#1}}% 
%
% norm
\newcommand{\norm}{\left\Vert \cdot \right\Vert}
\newcommand{\normof}[1]{\Vert #1 \Vert}
\newcommand{\normOf}[1]{\left\Vert #1 \right\Vert}
%
% ditributions
\newcommand{\unif}[1]{\mathsf{Unif}\brOf{#1}}
%
%
\newcommand{\equationFullstop}{\, .}
\newcommand{\eqfs}{\equationFullstop}
\newcommand{\equationComma}{\, ,}
\newcommand{\eqcm}{\equationComma}
\newcommand{\eqand}{\qquad\text{and}\qquad}
%
%
\newcommand{\sigmaAlg}{\mathrm{\sigma}}
\newcommand{\sigmaAlgof}[1]{\sigmaAlg(#1)}
\newcommand{\sigmaAlgOf}[1]{\sigmaAlg\brOf{#1}}
%
%
\newcommand{\euler}{\mathrm{e}}
%
%
%
\newcommand{\T}{^{T}}
%
%
%\newcommand*{\argmin}{\arg\min}
\DeclareMathOperator*{\argmin}{arg\,min}
\DeclareMathOperator*{\argmax}{arg\,max}
%
%
%
%
%
% Theorem, Lemma, Definition, Bemerkung
\def\betweenBoxHook{\vskip-2.0ex}
\def\postBoxSkip{1.0ex}
\def\postBoxSkipCmd{\vskip\postBoxSkip}
\def\preBoxSkip{1.0ex}
\def\preBoxSkipCmd{\vskip\preBoxSkip}
%
\declaretheoremstyle[
	bodyfont=\normalfont,
%	headformat={\S\NUMBER\ \NAME \NOTE},
	postfoothook={\postBoxSkipCmd},
	preheadhook={\preBoxSkipCmd},
	mdframed={
		backgroundcolor = black!2,
		startcode={\def\environmentEnumerateLabel{(\roman*)}},
	%	innerbottommargin = 10pt,
}]{ruledBoxStyle}
%
\declaretheoremstyle[
	bodyfont=\normalfont,
%	headformat={\S\NUMBER\ \NAME \NOTE},
	postfoothook={\postBoxSkipCmd},
	preheadhook={\preBoxSkipCmd},
	mdframed={
		backgroundcolor=white,
		%startcode={\def\environmentEnumerateLabel{(\roman*)}},
	%	innerbottommargin = 10pt,
}]{ruledBoxStyleWhite}
%
\declaretheoremstyle[
	bodyfont=\normalfont,
%	headformat={\S\NUMBER\ \NAME \NOTE},
	postfoothook={\postBoxSkipCmd},
	preheadhook={\preBoxSkipCmd},
	mdframed={
		backgroundcolor=black!2,
		linecolor = black!2,
		tikzsetting = {
			draw = black,
			line width = 2pt,%
			dashed,%
			dash pattern = on 10pt off 3pt
		},
}]{dashedBoxStyle}
%
\declaretheoremstyle[
	bodyfont=\normalfont,
%	headformat={\S\NUMBER\ \NAME \NOTE},
	postfoothook={\postBoxSkipCmd},
	preheadhook={\preBoxSkipCmd},
	mdframed={
		%backgroundcolor=black!2,
		linecolor = white,
		startcode={\def\environmentEnumerateLabel{(\roman*)}},
		tikzsetting = {
			draw = black,
			line width = 1pt,%
			loosely dotted,
			%dash pattern = on 1pt off 10pt,
		},
	}
]{dashedStyle}
%
\declaretheoremstyle[
	bodyfont=\normalfont,
	headformat={\NAME \NOTE},
	postfoothook={\postBoxSkipCmd},
	preheadhook={\preBoxSkipCmd},
	mdframed={
		%backgroundcolor=black!2,
		linecolor = white,
		startcode={\def\environmentEnumerateLabel{(\roman*)}},
		tikzsetting = {
			draw = black,
			line width = 1pt,%
			loosely dotted,
			%dash pattern = on 1pt off 10pt,
		},
	}
]{dashedStyle2}
%
\declaretheoremstyle[
	bodyfont=\normalfont,
%	headformat={\S\NUMBER\ \NAME \NOTE},
	postfoothook={\postBoxSkipCmd},
	preheadhook={\preBoxSkipCmd},
	mdframed={
		%backgroundcolor=black!2,
		linecolor = black,
		innerlinewidth=1pt,outerlinewidth=1pt,
		middlelinewidth=1pt,
		linecolor=black,middlelinecolor=white,
		startcode={\def\environmentEnumerateLabel{(\roman*)}},
	}
]{doubleStyle}
%
\declaretheoremstyle[
	bodyfont=\normalfont,
%	headformat={\S\NUMBER\ \NAME \NOTE},
	postfoothook={\postBoxSkipCmd},
	preheadhook={\preBoxSkipCmd},
	mdframed={
		backgroundcolor = black!4,
		linecolor = black!4,
		startcode={\def\environmentEnumerateLabel{(\alph*)}},
}]{boxStyle}
%
\declaretheoremstyle[
	headfont=\normalfont\itshape, 
	notefont=\normalfont\itshape, 
	notebraces={}{},
	bodyfont=\normalfont,
	qed=\qedsymbol,
	numbered=no,
%	headformat={\NAME},
	headindent=0pt,
	postheadspace=1ex,
	name={Proof},
	postheadhook={\def\environmentEnumerateLabel{(\roman*)}},
	mdframed={
		hidealllines = true,
		innerrightmargin = 0pt,
		innerleftmargin = 0pt,
		innertopmargin = 0pt,
		innerbottommargin = 0pt,
		leftmargin = 0pt,
		rightmargin = 0pt,
		%endinnercode={\qed}
	}
]{proofStyle}
%
\declaretheoremstyle[
	%headformat={\S\NUMBER\ \NAME \NOTE},
	bodyfont=\normalfont,
	postfoothook={\postBoxSkipCmd},
	preheadhook={\preBoxSkipCmd},
	mdframed={
		backgroundcolor = white,
		linecolor = black,
		startcode={\def\environmentEnumerateLabel{(\alph*)}},
		%		topline = false,
		%		bottomline = false,
		leftline = false,
		rightline = false,
}]{tobBottomStyle}
%
%
\declaretheoremstyle[
bodyfont=\normalfont,
% postfoothook={\vskip50pt},
% preheadshook={\preBoxSkipCmd},
%	mdframed={
%		innertopmargin = 5pt,
%		innerbottommargin = 5pt,
%		innerrightmargin = 10pt,
%		innerleftmargin = 10pt,
%		skipabove = 0pt,
%		%skipbelow = 0pt,
%		leftmargin = 0pt,
%		rightmargin = 0pt,
%		roundcorner = 0pt,
%		topline = true,
%		bottomline = true,
%		leftline = true,
%		rightline = true,
%		hidealllines = false,
%		linewidth = 0pt,
%		innerlinewidth = 0pt,
%		middlelinewidth = 0pt,	
%		outerlinewidth = 0pt,
%		backgroundcolor = white,
%		linecolor = white,
%		innerlinecolor = white,
%		middlelinecolor = white,	
%		outerlinecolor = white,
%		fontcolor = black,
%		font = ?,
%		shadow = false,
%		shadowsize = 8pt,
%		shadowcolor = black!50,
%		frametitle = {Title},
%		frametitle<Option...>,
%		%\mdfsubtitle{SubTitle}
%		subtitle<Option...>,
%		tikzsetting = {},
%		align =center,
%		extra = {},			
]{standardStyle}
%
%\declaretheorem[style=ruledBoxStyle,name=Definition,numberwithin=subsection]{definition}
\declaretheorem[style=ruledBoxStyle,name=Definition]{definition}
\declaretheorem[style=ruledBoxStyle,name=Lemma]{lemma}
\declaretheorem[style=ruledBoxStyle,name=Proposition]{proposition}
\declaretheorem[style=ruledBoxStyle,name=Theorem]{theorem}
\declaretheorem[style=ruledBoxStyle,name=Corollary]{corollary}
\declaretheorem[style=ruledBoxStyle,name=Theorem,numbered=no]{theorem*}
\declaretheorem[style=boxStyle,name=Remark]{remark}
\declaretheorem[style=boxStyle,name=Notation]{notation}
\declaretheorem[style=dashedStyle,name=Example]{example}
\declaretheorem[style=dashedStyle,name=Assumption]{assumption}
\declaretheorem[style=dashedStyle2,name=Assumptions]{assumptions}
\declaretheorem[style=dashedStyle,name=Conjecture]{conjecture}
\makeatletter
\let\proof\@undefined
\let\endproof\@undefined
\makeatother
\declaretheorem[style=proofStyle]{proof}
\def\theoremContentInNewLine{\text{}}
\def\environmentEnumerateLabel{(\roman*)}
%
%
%
%
\newcounter{subExample}%
\renewcommand{\thmcontinues}[1]{Teil \arabic{subExample}} % only for subExample
%
%
% ------------------------------------------------------------------------------
% ------------------------------------------------------------------------------
