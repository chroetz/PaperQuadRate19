\section{Conclusion}
%
For obtaining rates of convergence for the estimator of the generalized Fréchet mean, we introduced the quadruple inequality, which replaces a Lipschitz-con\-ti\-nu\-ity requirement of classical results. We showed that in some spaces, in particular in Hadamard spaces, this allows for better bounds without an otherwise common boundedness requirement.

We showed how three basic ingredients influence the obtained rate: the aforementioned quadruple inequality, a growth condition on the objective function, and an entropy condition.

We proved a power inequality, which is a special type of quadruple inequality that relates powers of distances in metric spaces. We applied it in the Fréchet mean setting, but it might be of interest on its own. 
%
%